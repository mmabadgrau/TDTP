\documentclass[runningheads]{/home/mabad/conferences/LNCS/llncs}

\usepackage{amssymb}
\setcounter{tocdepth}{3}
\usepackage{graphicx,multirow,hhline}
\usepackage{url}
\newcommand{\keywords}[1]{\par\addvspace\baselineskip
\noindent\keywordname\enspace\ignorespaces#1}

\newcommand{\vs}[0]{\vspace{-4.2 mm}}

\begin{document}

\mainmatter 


\title{Configurations used with program msHOT}


\titlerunning{Simulations}



\author{Mar\'{\i}a M. Abad-Grau\inst{1}
\thanks{Corresponding author: mabad@ugr.es}, Nuria Medina-Medina\inst{1}, Rosana Montes\inst{1} and Fuencisla Matesanz \inst{2}}
%



\authorrunning{Abad-Grau et al.}   % abbreviated author list (for running head)

\institute{Departamento de Lenguajes y Sistemas Inform\'aticos \\Universidad de Granada, Granada 18071, Spain\\
\and Instituto de Parasitolog{\'i}a y Biomedicina L\'opez Neyra \\
Consejo Superior de Investigaciones Cient{\'i}ficas \\
Granada, Spain \\
}

\date{}

\maketitle


\section{Introduction}\label{sec:introduction}

Recombination fraction within blocks for Africans is $0.000088$, which is equivalent to a genetic distance of $0.0088 cM$ or a physical distance of $8800 bp$. Recombination fraction within blocks for Europeans is $0.000207$, which is equivalent to a genetic distance of $0.0207 cM$ or a physical distance of $20700 bp$. We used Africans. 

\subsection{Population evolution:}
In msHOT evolution is computed backwards from present time.
\subsubsection{Africans}
We want to model a populations which had constant size $N_1=10000$ 10000 generations ago and in the next 10000 generations had exponential growth to have a final size $N_2=N_0=1000000$ individuals.
\begin{enumerate}
\item Exponential growth: The populations had an exponential growth during $ng=10000$ generations starting at $N_1=10000$ individuals and finishing with $N_2=N_0=1000000$ individuals. This is configured as $-G$ $\alpha$, where $\alpha$ is computed by solving equation:
\[N_1=N_2 exp^{-\alpha ng/(4 N_0)},\]
which is $\alpha=1842.06807$. $-G$ $\alpha$ sets growth parameter to $\alpha$. 
\item Constant size: Before that exponential growth, the population was at constant size, thus it was at constant size since 10000 generations ago backwards, which is $eG$ $ng/4$ $N_0=0.0025$  $0.0$
\end{enumerate}

\subsubsection{Europeans}
We want to model a populations which had constant size $N_1=10000$ 10000 generations ago, then it had an instanteneous size change (bottle neck) and shrinked to $N_2=1000$ individuals and in the next 10000 generations had exponential growth to have a final size $N_3=N_0=1000000$ individuals.
\begin{enumerate}
\item Exponential growth: The populations had an exponential growth during $ng=10000$ generations starting at $N_1=1000$ individuals and finishing with $N_3=N_0=1000000$ individuals. This is configured as $-G$ $\alpha$, where $\alpha$ is computed by solving equation:
\[N_1=N_3 exp^{-\alpha ng/(4 N_0)},\]
which is $\alpha=2763.1021$. $-G$ $\alpha$ sets growth parameter to $\alpha$. 
\item Constant size: Before that exponential growth, the population was at constant size, thus it was at constant size 10000 generations ago backwards, which is $eG$ $ng/4$ $N_0=0.0025$  $0.0$
\item Bottle neck: At generation $ng=10000$ it had an instantaneous size change from 10000 individuals to 1000. This is configured with parameter $-eN$ $ng/(4 N_0)$ $s$, to assign a size of $s\times N_0$ individuals at generation $ng$ (in units of $4N_0$) and a growth rate of $0$. Thus, it will be $-eN$ $0.0025$ $0.01$. 
\end{enumerate}

\subsection{Number of sites}
We want to simulate a block of low recombination plus two disease loci, one at the left and the other at the right of the block. 
We will use 1000 SNPs for each disease locus, in order to increase the chances of having at least 10 SNPs for each disease locus within the frequency interval chosen for the MAF of the disease allele.
The number of sites for the block is the number of base pairs, which depends on the recombination factor in the block and thus on the populations. Thus, the total number of sites is $10800$ and $22700$ for African and European respectively. 

\subsection{Mutation rate}

Mutation rate is $10e-8$ for both populations. Thus, for all the base pairs is $-t$ $4NxtotalBasePairsx10e-8$, which is  432 for Africans and 908 for Europeans. 

\subsection{Basic recombination rate}
Recombination rate within the block will be set by using $-r$ $4N_0r\times totalSites$ $totalSites$, which $r$ being the probability of cross-over per generation between the end of the locus being simulated. We will consider $r'$, recombination between consecutive base pairs, to be $10e-8$ and thus $r=10e-8 \times totalBasePairs$. Therefore, it will be $-r$  $432$ $10800$ and $-r$ $908$ $22700$ for Africans and Eurpeans respectively.

\subsection{Recombination rates to disease loci}
Recombination rate to disease loci is configured as two hotspots in the following
way:

$-v$ $2$  $1000$ $1001$ $recBetween/recPerSiteWithinBlock$ [$nsites-1000$] [$nsites-999$] $recBetween2/recWithinBlock$


where $recBetween$ will change to consider different recombination rates to a disease loci. 
It follows that normal recombination rate $recPerSiteWithinBlock$ was set to $10e-8$, so $recBetween/recPerSiteWithinBlock$ is computed as $recBetween*10e8$. $recBetween2$ is fixed to $0.1$, in order to simulate two disase locus not in linkage.





\end{document}

